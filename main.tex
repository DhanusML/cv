%%%%%%%%%%%%%%%%%%%%%%%%%%%%%%%%%%%%%%%%%
% Medium Length Graduate Curriculum Vitae
% LaTeX Template
% Version 1.1 (9/12/12)
%
% This template has been downloaded from:
% http://www.LaTeXTemplates.com
%
% Original author:
% Rensselaer Polytechnic Institute (http://www.rpi.edu/dept/arc/training/latex/resumes/)
%
% Important note:
% This template requires the res.cls file to be in the same directory as the
% .tex file. The res.cls file provides the resume style used for structuring the
% document.
%
%%%%%%%%%%%%%%%%%%%%%%%%%%%%%%%%%%%%%%%%%

%----------------------------------------------------------------------------------------
%	PACKAGES AND OTHER DOCUMENT CONFIGURATIONS
%----------------------------------------------------------------------------------------

\documentclass[margin, 10pt]{res} % Use the res.cls style, the font size can be changed to 11pt or 12pt here

\usepackage{helvet} % Default font is the helvetica postscript font
%\usepackage{newcent} % To change the default font to the new century schoolbook postscript font uncomment this line and comment the one above

\usepackage{hyperref}
\setlength{\textwidth}{5.1in} % Text width of the document

\begin{document}

%----------------------------------------------------------------------------------------
%	NAME AND ADDRESS SECTION
%----------------------------------------------------------------------------------------

\moveleft.5\hoffset\centerline{\Large\bf Dhanus M Lal} % Your name at the top
 
\moveleft\hoffset\vbox{\hrule width\resumewidth height 1pt}\smallskip % Horizontal line after name; adjust line thickness by changing the '1pt'
 
\moveleft.5\hoffset\centerline{Indian Institute of Science, Bengaluru, Karnataka}
\moveleft.5\hoffset\centerline{Phone: +91-9746949506} % Your address
%\vspace{3mm}\\
\moveleft.5\hoffset\centerline{Email: \href{dhanusmlal@gmail.com}{dhanusmlal@gmail.com} \vrule \, IISc Email:
\href{dhanuslal@iisc.ac.in}{dhanuslal@iisc.ac.in}}
\moveleft.5\hoffset\centerline{Linkedin: \href{https://www.linkedin.com/in/dhanusmlal}{DhanusMLal}
\vrule \, Github : \href{https://github.com/dhanusml}{DhanusML}
\vrule \, Skype: live:mlmanikandan}


%----------------------------------------------------------------------------------------

\begin{resume}

%----------------------------------------------------------------------------------------
%	OBJECTIVE SECTION
%----------------------------------------------------------------------------------------
 
\section{SUMMARY}  

	I am an MSc(Research) mathematics student at IISc, Bangalore.
	I am looking for internships or placements in areas related to
	mathematics/data science.

%----------------------------------------------------------------------------------------
%	EDUCATION SECTION
%----------------------------------------------------------------------------------------

\section{EDUCATION}
{\sl Bachelor of Science (Research)} \hfill CGPA - 8.8\\
Indian Institute of Science, Bengaluru\\
Major: Mathematics\\

{\sl Higher secondary}[ISC]  \hfill  96\%\\
St. John's Residential School, Kollam, Kerala

{\sl High school}[CBSE] \hfill CGPA - 10\\
City Central School, Kollam, Kerala 

%----------------------------------------------------------------------------------------
%	COMPUTER SKILLS SECTION
%----------------------------------------------------------------------------------------

\section{SKILLS} 

{\sl Programming Languages \& Packages:} 
 C, C++, Python, MATLAB, \LaTeX,
	NumPy, SciPy, PyTorch, MPI, OpenMP, Git.

{\sl Operating Systems:} Windows, Linux.

{\sl Mathematical skills:} Very strong mathematical background,
	Linear Algebra, Measure Theory, Probability Theory,
	Machine Learing, Sparse Recovery.

{\sl Other skills:} Typing speed: 65 WPM
 
%----------------------------------------------------------------------------------------
%	PROFESSIONAL EXPERIENCE SECTION
%----------------------------------------------------------------------------------------
 
\section{PROJECTS}

{\sl Thesis project on compressed sensing} \hfill  Jan -- May 2022\\
Advisor: Manjunath Krishnapur, Department of Mathematics, IISc, Bengaluru.
\begin{itemize} \itemsep -2pt
	\item Techniques for recovering sparse signal from a linear measurement.
	\item Explored how geometry of the set of sparse vectors
		guarantee exact recovery using basis-pursuit.
\end{itemize} 

\smallskip

{\sl Dimensionality reduction: Machine learning course project} \hfill Mar -- May 2022\\
	Advisor: Chaitanya Murti, CSA Department, IISc, Bengaluru.
	\begin{itemize}
		\item Analyzed and implemented various dimensionality
			reduction techniques on CIFAR-10 dataset.
		\item Performance of each method was compared using various
			linear and non-linear classifiers.
		\item Dimensionality reduction methods studied: PCA,
			kernel-PCA, linear discriminant analysis, autoencoders
			and Johnson Lindenstrauss lemma.
	\end{itemize}

\smallskip

{\sl Reading project on Zorn's lemma} \hfill Jul -- Aug 2021 \\
Advisor: Arvind Ayyer, Department of Mathematics, IISc, Bengaluru.
\begin{itemize} 
	\item Explored the equivalence between Zorn's lemma,
		axiom of choice and well ordering principle.
\end{itemize} 

\smallskip

{\sl Modelling Bernoulli bond percolation in 2-dimensional lattice}\hfill Jun -- Oct 2020\\
Github repository is linked
\href{https://github.com/AbhinavM2000/percolation\_}{here}
\begin{itemize}
	\item Used C and python to estimate percolation threshold in a 2D lattice
\end{itemize}

\smallskip

{\sl Developed a simple board game using python}\hfill May -- Jun 2020\\
Github repository is linked
\href{https://github.com/DhanusML/marble-and-hole-puzzle}{here}.

\smallskip

{\sl Reading project on probability thoery} \hfill Jun -- Aug 2019 \\
Advisor: Arvind Ayyer, Department of Mathematics, IISc, Bangalore
\begin{itemize} \itemsep -2pt % Reduce space between items
	\item Explored elementary topics in probability theory.
	\item Baye's theorem, conditional probabilities,
		conditional expectation, etc.
\end{itemize}




\section{ACHIEVMENTS}

{\sl Keysight IoT challenge 2019 entry accepted}\\
Distributed Real-Time Air Quality Indexing System concept accepted
as an entry in the smart land category of Keysight IoT challenge
(linked \href{https://www.iotchallengekeysight.com/2019/entries/smart-land/41-0413-102730-draqis-distributed-real-time-air-quality-indexing-system}{here})

{\sl Kishore Vaigyanik Protsahan Yojana (KVPY) fellow}\\
Qualified KVPY exam in 2018 with all india rank 61.


%----------------------------------------------------------------------------------------
%	EXTRA-CURRICULAR ACTIVITIES SECTION
%----------------------------------------------------------------------------------------


\section{OTHER \\ ACTIVITIES} 
{\it Performed Belousov-Zhabotinsky reaction} as an exhibit
in the UG chemistry lab during open day IISc, 2020

\smallskip

Volunteered for various events in the UG cultural and
tech fest {\it Pravega 2018 and 2019} at IISc.

\smallskip

Represented City Central School in {\it South Zone Sahodaya sports meet}
for the events long jump, 100m sprint and 4$\times$100m relay.

%----------------------------------------------------------------------------------------

\end{resume}
\end{document}
